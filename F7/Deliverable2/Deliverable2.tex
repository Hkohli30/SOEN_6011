\documentclass[a4paper, 11pt]{article}
\usepackage{comment} % enables the use of multi-line comments (\ifx \fi) 
\usepackage{fullpage} % changes the margin
\usepackage{hyperref}
\usepackage{mathtools}
\usepackage{amsmath}
\usepackage{amssymb}
\usepackage{booktabs} % For formal tables
\usepackage[ruled]{algorithm2e} % For algorithms
\renewcommand{\algorithmcfname}{ALGORITHM}
\DeclarePairedDelimiter{\ceil}{\lceil}{\rceil}

\begin{document}

\begin{titlepage}
   \begin{center}
       \vspace*{1cm}
 
       \textbf{\huge{SOEN 6011: SOFTWARE ENGINEERING PROCESSES}}
 
       \vspace{0.5cm}
        \textbf{\huge{Project Deliverable 2}}
 
       \vspace{1.0cm}
       \vskip 1.4in
    \includegraphics[width=1.0\textwidth]{Concordia.jpg}
       
       \vskip 1.4in
 
       \vspace{0.8cm}
       
        \textbf{Himanshu Kohli}\\
        \textbf{Student Id : 40070839}\\
       Applied Computer Science\\
       Concordia University\\
       
 
   \end{center}
\end{titlepage}

% @author : Himanshu Kohli
% Student ID: 40070839
% Date: 22 Feb 2019
\noindent
\large\textbf{Problem 4,5,6} \hfill \textbf{Himanshu Kohli} \\
\normalsize SOEN 6011 \hfill \textbf{40070839} \\
Prof. Pankaj Kamthan \hfill Due Date: July 26, 2019 \\


\section{Debugger}
\subsection{Introduction}
Debugging is a process of locating and removing bugs,errors and abnormalities form a programs. Bugs are subtle conditions which leads inefficient programs and can only be found when specific conditions occurs.
\subsection{Debugger Used}
The Eclipse IDE provides a bug finding tool which is combined in the views as Debug perspective which helps us to view execution of the program and the variables/constant at every stage of the program. Eclipse IDE Debugger is an source-level debugger or symbolic debugger which finds traps(conditions which doesn't allow the program to execute normally) in the system. Eclipse inbuilt debgger has many features which helps such as:
\begin{enumerate}
    \item Single Stepping : A step by step execution of program helps the programmer understand, see and find the error by viewing the values of all the variables.
    \item Breakpoints: Pausing the state of the program for examination 
    \item Tracepoints : Allows user to create conditional breakpoints to print messages without halting the breakpoints
\end{enumerate}
\subsection{Advantages and Disadvantages}
\subsubsection{Advantages}
\begin{enumerate}
    \item \textbf{Watches} in debugging eliminates the need to add additional print line statements to the source code and removing it after finding the bug.
    \item It removes the tedious and time consuming process of rebuilding and executing the program while at most times \textbf{hot swapping} the changes.
    \item Stepping in and out of some part of the code saves time for the user and increase programmer's efficiency to find bugs.
\end{enumerate}

\subsubsection{Disadvantages}
\begin{enumerate}
    \item For a smaller code which can be understood by print line statement it can be rather time consuming to use debugger
    \item Overloads the system's resources in getting value of all variables at execution time.
\end{enumerate}


\newpage
\section{Code Quality Checking Tool }
\subsection{Introduction}
\subsection{Tool Used}
\subsection{Advantages and Disadvantages}
\begin{thebibliography}{}
\bibitem{wikipeda} 
Debugger
\url{https://en.wikipedia.org/wiki/Debugger}

\bibitem{eclipse}
Debugging the Eclipse IDE for Java Developers | The Eclipse Foundation
\url{https://www.eclipse.org/community/eclipse_newsletter/2017/june/article1.php}
\bibitem{stackDebug}
What are the advantages of using the Java debugger over println?
\url{https://softwareengineering.stackexchange.com/questions/168540/what-are-the-advantages-of-using-}

\end{thebibliography}

\end{document}